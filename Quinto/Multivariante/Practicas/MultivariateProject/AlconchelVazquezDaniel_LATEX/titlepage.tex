\begin{titlepage}
    \begin{center}
    {\fontsize{40}{48}\selectfont \bfseries Multivariate Statistics} 
    \\\vspace{20pt}
    {\LARGE Final assigment} \\
    \vspace{20pt}
    \textbf{Daniel Alconchel, Mario García, Pablo Fuentes}
    \vspace{8pt}
    \\ \today
    \end{center}

    \bigskip
\begin{abstract}
In this project, we will analyze a database containing data on various aspects of residential homes in Ames, Iowa.

Our initial step involves a comprehensive exploratory data analysis to identify potential missing values and outliers. We will make decisions to address these issues.

Secondly, we will conduct a Principal Component Analysis (PCA). This technique aims to condense information from the original variables into a few linear combinations. The objective is to achieve dimensionality reduction while maximizing variance. These linear combinations are designed to be perpendicular to each other, aligning with the directions of maximum variance and ensuring lack of correlation.

Next, we will perform Factor Analysis (FA), identifying latent variables that exhibit a high correlation with specific groups of observable variables and minimal correlation with others. FA facilitates dimensionality reduction by capturing the underlying structure in the data.

In the final stage, we will execute both Linear Discriminant Analysis (LDA) and Quadratic Discriminant Analysis (QDA). Prior to these analyses, we will verify the necessary assumptions of normality. Discriminant Analysis is a classification method for qualitative variables. It allows the categorization of new observations based on their characteristics (explanatory or predictor variables) into different categories of the qualitative response variable
    \end{abstract}
\end{titlepage}