% Preámbulo

% Tipo de documento
\documentclass[a4paper, 12pt]{article}
% Codificación
\usepackage[utf8]{inputenc} 
\usepackage{fancyhdr}
\usepackage{graphicx}
% Idioma
\usepackage[spanish]{babel}
%Tipografía
\usepackage{gfsartemisia}
\usepackage[T1]{fontenc}
\usepackage[usenames]{color}
\definecolor{Lila}{rgb}{0.39,0.17,0.63}
\definecolor{Azul}{rgb}{0.17,0.39,0.63}
%%% Matemáticas Paquetes AMS
\usepackage{amsfonts}
\usepackage{amsmath} % Matemáticas
\usepackage{amssymb,amsthm} % Símbolos y teoremas
\usepackage{mathtools} % % Algunos añadidos y correcciones a amsmath
\usepackage{esvect}
%referencia
\usepackage{hyperref}
% Título y autor de un documento. Fecha. Tabla de contenido
\pagestyle{fancy}
\fancyhf{}
\rhead{Ejercicio de Entrega 1}
\lhead{Daniel Alconchel Vázquez}
\rfoot{Page \thepage}

%Maths
\theoremstyle{plain}
\newtheorem{teorema}{Teorema}[section]
\newtheorem{coro}[teorema]{Corolario}
\newtheorem{proposicion}[teorema]{Proposición}
\newtheorem{lema}[teorema]{Lema}
\theoremstyle{definition}
\newtheorem{definicion}[teorema]{Definición}
\theoremstyle{remark}
\newtheorem*{observacion}{Observación}

\begin{document}
	\begin{flushleft}
		\textbf{1.} Se denota por $(\Omega,\mathcal{A},P)$ el espacio probabilístico base. Se considera la siguiente definición de medida de probabilidad:
		\begin{definicion}
			$P:\mathcal{A}\rightarrow[0,1]$, es una función de probabilidad si satisface los siguientes tres axiomas:
			\begin{enumerate}
				\item\label{1} $P(A)\geq 0, \quad \forall A \in \mathcal{A}$
				\item\label{2} $P(\Omega)=1$
				\item\label{3} Para cualquier secuencia $\{A_n\}_{n\in \mathbb{N}}\subseteq \mathcal{A}$ de sucesos disjuntos
				\[P\left(\bigcup_{n\in \mathbb{N}}A_n\right)=\sum_{n\in \mathbb{N}}P(A_n)\]
			\end{enumerate}
		\end{definicion}
		Demostrar a partir de la definición, las siguientes propiedades:
		\renewcommand{\theenumi}{\alph{enumi}}
		\begin{enumerate}
			\item\label{a} $P(\emptyset) = 0$
			\item\label{b} Probabilidad del suceso complementario: $P(A^c)=1-P(A)$
			\item\label{c} Aditividad finita para procesos disjuntos: $P(\bigcup_{n=1}^{N}A_n)=\sum_{n=1}^{N}P(A_n)$
			\item\label{d} Probabilidad de la diferencia y la monotonía: $B\subseteq A\in \mathcal{A}, P(A-B)=P(A)-P(B), P(B)\leq P(A)$
			\item\label{e} $A,B \in \mathcal{A}, P(A\cup B)=P(A)+P(B)-P(A\cap B)$
			\item\label{f} Principio de inclusión-exclusión para la unión finita de sucesos no disjuntos.
			\item\label{g} Subaditividad:$P(\bigcup_{n\in \mathbb{N}}A_n)\leq\sum_{n\in \mathbb{N}}P(A_n)$
			\item\label{h} Desigualdad de Boole: $P(\bigcap_{n\in \mathbb{N}}A_n)\geq 1-\sum_{n\in \mathbb{N}}P(A^c)$
		\end{enumerate}
	\end{flushleft}
	\newpage
	\begin{itemize}
		\item Comenzaremos demostrando \ref{b}. Para ello usaremos los axiomas \ref{2} y \ref{3}.Usando que $A\cup A^c=\Omega, \forall A\in \mathcal{A}$, se extrae que:
		\[P(A^c)+P(A) \overset{\ref{3}}{\implies} P(A^c)=P(\Omega)-P(A)\overset{\ref{2}}{\implies} P(A^c)=1-P(A)\]
		$\hfill\square$
		\item Ahora estamos en disposición de demostrar \ref{a}. Para ello usaremos \ref{b}, el cual, acabamos de demostrar y el axioma \ref{2}.
		\[P(\Omega^c)=P(\emptyset) \overset{\ref{b}}{=} 1-P(\Omega) \overset{\ref{2}}{=} 0 \]
		$\hfill\square$
		\item Vamos a demostrar ahora \ref{c}. Para ello consideramos una sucesión $\{A_n\}_{n\in \mathbb{N}}$ de sucesos disjuntos, de  forma que de un valor $n_i \in \mathbb{N}$ en adelante, los elementos de la sucesión sean el conjunto vacío, es decir:
		\[
		\left\{ \begin{array}{lcc}
		X_n=A_i \quad \forall n \in \mathbb{N}, \quad 1\leq n \leq n_i \\
		\\ X_n=\emptyset \quad \forall n \in \mathbb{N}, \quad n \geq n_i   \\
		\end{array}\]
		Usando esto, \ref{3} y lo ya demostrado (\ref{a}), obtenemos:
		\[P\left(\bigcup_{n=1}^{\infty} X_n\right)\overset{\ref{3}}{=}\sum_{n=1}^{\infty}P(X_n)=\sum_{n=1}^{n_i}P(X_n)+\sum_{n_i}^{\infty}P(\emptyset)\overset{\ref{a}}{=}\sum_{n=1}^{n_i}P(X_n)=P\left(\bigcup_{n=1}^{n_i}X_n\right)
		\]
		Como $n_i\in \mathbb{N}$ no está fijo, esto es válido para cualquier $n_i\in \mathbb{N}$.
		
		$\hfill\square$
		\item Continuamos demostrando \ref{d}. Para ello tomamos $B\subseteq A\in \mathcal{A}\implies A\cap B = B$, con lo que podemos realizar el siguiente desarrollo:
		\[A-B=A-(A\cap B) \implies A=(A-B)\cup (A\cap B)
		\]
		De donde se extrae que:
		\[P(A)=P(A-B)+P(A\cap B)=P(A-B)+P(B) \implies \]
		\[\implies P(A-B)=P(A)-P(B) \implies P(B)=P(A)-P(A-B) \overset{\ref{1}}{\implies} P(B) \leq P(A)
		\]
		
		$\hfill\square$
		\item Para demostrar \ref{e} vamos a tomar $A,B\in \mathcal{A}$ y consideramos la unión de dichos conjuntos, la cual, podemos expresar como una partición de la siguiente forma: $A\cup B=(A-B)\cup(A\cap B)\cup (B-A)$. De esta forma obtenemos:
		\[P(A\cup B)\overset{\ref{c}}{=}P(A-B)+P(A\cap B)+P(B-A)=
		\]
		\[=P(A)-P(A\cap B)+P(A\cap B)+P(B)-P(A\cap B)=P(A)+P(B)-P(A\cap B)
		\]
		
		$\hfill\square$
		\item Demostrar \ref{f} equivale a demostrar que: 
		
		\[P(\bigcup_{n=1}^N A_n)= \sum_{n=1}^{N}P(A_n)-\sum_{i_1,i_2=1.i_!<i_2}^{N}P(A_{i_1}\cap A_{i_2})+\sum_{i_1,i_2,i_3=1,i_1<i_2<i_3}^{N}P(A_{i_1}\cap A_{i_2}\cap A_{i_3})+...\]
		\[...+(-1)^{N+1}P(\bigcap_{i=1}^N A_i), \qquad A_n\in\mathcal{A},\forall n\in {1,...,N}\]
		
		lo cual probaremos por inducción:
		\begin{itemize}
			\item Para N=2 tenemos que se trata de lo demostrado en \ref{e}.
			\item Supongamos cierto para N y demostremos que es cierto para N+1:
			\[
			P\left(\bigcup_{i=1}^{N+1} A_i\right)=P\left(\left(\bigcup_{i=1}^N A_i\right)\cup A_{N+1}\right)=P\left(\bigcup_{i=1}^N A_i\right)+P(A_{N+1})-P\left(\left(\bigcup_{i=1}^N A_i\right)\cap A_{N+1}\right)
			\]
			Ahora, aplicando la propiedad distributiva de la unión respecto de la intersección en el último sumando se tiene:
			\[
			P\left(\bigcup_{i=1}^{N+1} A_i\right)=P\left(\bigcup_{i=1}^N A_i\right)+P(A_{N+1})-P\left(\bigcup_{i=1}^N (A_i\cap A_{N+1})\right)
			\]
			A continuación, ya que la propiedad se está suponiendo cierta para la unión de N sucesos, se aplica al primer y tercer sumando de la expresión y se tiene que:
			\[
			\begin{split}
			P\left(\bigcup_{i=1}^{N+1} A_ i\right)= \\
			= \left[\sum_{n=1}^{N}P(A_n)-\sum_{i_1,i_2=1.i_!<i_2}^{N}P(A_{i_1}\cap A_{i_2})+\sum_{i_1,i_2,i_3=1,i_1<i_2<i_3}^{N}P(A_{i_1}\cap A_{i_2}\cap A_{i_3})+... \\
			...+(-1)^{N+1}P(\bigcap_{i=1}^N A_i)\right]+P(A_{N+1})-\\
			-\left[\sum_{i_1=1}^{N}P(A_{i_1}\cap A_{N*1})-\sum_{i_1,i_2=1,i_1<i_2}^N P(A_{i_1}\cap A_{i_2}\cap A_{N+1})+...+(-1)^{N+1}P\left(\bigcap_{i=1}^{N+1}A_i\right)\right]
			\end{split}
			\]
			Observe que en el segundo corchete aparecen las sumas de las probabilidades de intersecciones dos a dos, tres a tres, etc...,que no aparecen en el primer corchete, esto es, las probabilidades de las intersecciones $A_{N+1}$ con el resto de sucesos, por tanto, redondeando las sumas, se tiene que la propiedad para la unión de N+1 sucesos es:
			\[
			\begin{split}
			P\left(\bigcup_{i=1}^{N+1}A_i\right)=\\
			\sum_{n=1}^{N+1}P(A_n)-\sum_{i_1,i_2=1.i_!<i_2}^{N+1}P(A_{i_1}\cap A_{i_2})+\sum_{i_1,i_2,i_3=1,i_1<i_2<i_3}^{N+1}P(A_{i_1}\cap A_{i_2}\cap A_{i_3})+...\\
			...+(-1)^{N+2}P(\bigcap_{i=1}^{N+1} A_i)
			\end{split}
			\]
		\end{itemize}
	
		$\hfill\square$
		\item Para demostrar \ref{g} tomamos una colección de eventos $\{A_n\}_{n\in\mathbb{N}}$ y definimos:
		\[
		\left\{ \begin{array}{lcc}
		X_1=A_1 \\
		\\ X_n=A_n-\bigcup_{k=1}^{n-1}A_k \quad n=2,3...  \\
		\end{array}\]
		De esta forma, tenemos una colección de eventos disjuntos dos a dos, que verifica:
			\begin{itemize}
				\item $\bigcup_{n=1}^{\infty}A_n=\bigcup_{n=1}^{\infty}X_n$
				\item $X_n\subseteq A_n$
			\end{itemize}
		Por lo tanto:
		\[
		P\left(\bigcup_{n=1}^\infty A_n\right)=P\left(\bigcup_{n=1}^\infty X_n\right)\overset{\ref{3}}{=}\sum_{n=1}^{\infty}P(X_n)\overset{X_n\subseteq A_n}{\leq}\sum_{n=1}^{\infty}P(A_n)
		\]
		
		$\hfill\square$
		\item Por último, para probar \ref{h}, tomamos una colección de eventos $\{A_n\}_{n\in\mathbb{N}}$ y usando álgebra de Boole llegamos a:
		\[
		P\left(\bigcap_{n=1}^{\infty}A_n\right)=1-P\left(\bigcup_{n=1}^\infty A_n^c\right)\overset{\ref{g}}{\geq}1-\sum_{n=1}^\infty P(A_n^c)
		\]
		
		$\hfill\square$
	\end{itemize}
\end{document}