% Preámbulo

% Tipo de documento
\documentclass[a4paper, 12pt]{article}
% Codificación
\usepackage[utf8]{inputenc} 
\usepackage{fancyhdr}
\usepackage{graphicx}
% Idioma
\usepackage[spanish]{babel}
%Tipografía
\usepackage{gfsartemisia}
\usepackage[T1]{fontenc}
\usepackage[usenames]{color}
\definecolor{Lila}{rgb}{0.39,0.17,0.63}
\definecolor{Azul}{rgb}{0.17,0.39,0.63}
%%% Matemáticas Paquetes AMS
\usepackage{amsfonts}
\usepackage{amsmath} % Matemáticas
\usepackage{amssymb,amsthm} % Símbolos y teoremas
\usepackage{mathtools} % % Algunos añadidos y correcciones a amsmath
\usepackage{esvect}
%referencia
\usepackage{hyperref}
% Título y autor de un documento. Fecha. Tabla de contenido
\pagestyle{fancy}
\fancyhf{}
\rhead{Ejercicio de Entrega 4}
\lhead{Daniel Alconchel Vázquez}
\rfoot{Page \thepage}

%Maths
\theoremstyle{plain}
\newtheorem{teorema}{Teorema}[section]
\newtheorem{coro}[teorema]{Corolario}
\newtheorem{proposicion}[teorema]{Proposición}
\newtheorem{lema}[teorema]{Lema}
\theoremstyle{definition}
\newtheorem{definicion}[teorema]{Definición}
\theoremstyle{remark}
\newtheorem*{observacion}{Observación}

\begin{document}
	Completar las siguientes afirmaciones:
	\begin{flushleft}
		\begin{enumerate}
			\item Sean $X_1,...,X_m$ son independientes si y sólo sí, para cualquiera $B_i \in \mathcal{B}^{n_i},i=1,...,m$ se tienen las siguientes identidades:
				\[
				P_{X_1,...,X_m}(B_1\times...\times B_m)=P(X_1\in B_1,...,X_n \in B_n)=P_{X_1}(B_1)...P_{X_n}(B_n)=P(X_1\in B_1)...P(X_n\in B_n)
				\]
			\item Supongamos que existen las funciones generatrices de momentos de $X_1,...,X_n$, respectivamente, definidas en los intervalos $I_i=\prod_{j=1}^{n_i}(-a_{j,i},b_{j,i}),a_{j,i},b_{j,i}>0,I_i\subseteq R^{n_i},i=1,...,m$. Entonces, $X_1,...,X_m$ son independientes si y sólo sí, para cualesquiera $(t_1,...,t_m)\in I_1\times ...\times I_m$, se tiene:
			\[
			M_{X_1,...,X_m}(t_1,...,t_m)=M_{X_1}(t_1)...M_{X_m}(t_m)
			\]
			\item Si $X_1,...,X_m$ son independientes, cualquier subconjunto $X_{i_1},...,X_{i_k},0<k<n$, de $X_1,...,X_m$ \textbf{son también independientes}.
			\item Si $X_1,...,X_m$, son variables aleatorios independientes, para cualesquiera $g_i, i=1,...,m$, aplicaciones medibles, las variables aleatorias $g_1(X_1),...,g_m(X_m)$ \textbf{también son independientes}.
		\end{enumerate}
	\end{flushleft}
\end{document}