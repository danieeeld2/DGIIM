% Preámbulo

% Tipo de documento
\documentclass[a4paper, 12pt]{article}
% Codificación
\usepackage[utf8]{inputenc} 
% Idioma
\usepackage[spanish]{babel}
%Tipografía
\usepackage{gfsartemisia}
\usepackage[T1]{fontenc}
\usepackage[usenames]{color}
\definecolor{Lila}{rgb}{0.39,0.17,0.63}
\definecolor{Azul}{rgb}{0.17,0.39,0.63}
%%% Matemáticas Paquetes AMS
\usepackage{amsfonts}
\usepackage{amsmath} % Matemáticas
\usepackage{amssymb,amsthm} % Símbolos y teoremas
\usepackage{mathtools} % % Algunos añadidos y correcciones a amsmath
\usepackage{esvect}
%referencia
\usepackage{hyperref}
% Título y autor de un documento. Fecha. Tabla de contenido
\title{La Eficiencia de los Algoritmos}
\author{Daniel Alconchel Vázquez}
\date{25 de marzo de 2021}

%Maths
\theoremstyle{plain}
\newtheorem{teorema}{}[section]
\newtheorem{coro}[teorema]{Corolario}
\newtheorem{proposicion}[teorema]{Proposición}
\newtheorem{lema}[teorema]{Lema}
\theoremstyle{definition}
\newtheorem{definicion}[teorema]{Definición}
\theoremstyle{remark}
\newtheorem*{observacion}{Observación}

\begin{document}
	\maketitle
	\newpage
	\tableofcontents
	\newpage
	
	\section{Planteamiento general}
	Una definición formal y general de \textbf{algoritmo} es una secuencia finita y ordenada de pasos, extensos de ambigüedad, que dará como resultado que se realice la tarea para la que se ha diseñado con recursos limitados y en tiempo finito.
	
	\begin{definicion}\label{Bazaraa}
		\textbf{Definición de Bazaraa, Sheraly y Shetty:}
		
		Un algoritmo para resolver un problema es un proceso iterativo que genera una sucesión de puntos, conforme a un conjunto dado de instrucciones y con un criterio de parada.
	\end{definicion}

	\medskip
	Un algoritmo tiene cinco características primordiales:
	\begin{itemize}
		\item \textbf{Finitud}: ha de terminar después de un tiempo acotado.
		\item \textbf{Especificidad}: cada etapa debe estar precisamente definida.
		\item \textbf{Input}: un algoritmo tiene cero o más inputs.
		\item \textbf{Output}: uno o más outputs.
		\item \textbf{Efectividad}: todas las operaciones deben ser tan básicas como para que se puedan realizar en un periodo finito de tiempo.
	\end{itemize}

	\section{Tiempo de ejecución. Notaciones para la eficiencia de los algoritmos}
	\begin{definicion}
		\textbf{Principio de Invariancia:}
		
		Dos implementaciones diferentes de un mismo algoritmo no difieren en eficiencia más que, a lo sumo, en una constante multiplicativa.
	\end{definicion}

	Parece oportuno referirnos a la eficiencia teórica de un algoritmo en términos de tiempo. Algo que conocemos de antemano es el denominado \textbf{Tiempo de Ejecución} de un programa, que depende de:
	\begin{itemize}
		\item Input del programa (Depende de su tamaño).
		\item La calidad del código que genera el compilador.
		\item Naturaleza y velocidad de las instrucciones máquina.
		\item Complejidad en tiempo en tiempo del algoritmo.
	\end{itemize}

	\textcolor{Azul}{Usamos la notación \textbf{T(n)} para referirnos al tiempo de ejecución de un programa para un input de tamaño n.}
	
	\bigskip
	Diremos que un algoritmo consume un tiempo de orden t(n), si existe una constante positiva, c, y una implementación del algoritmo capaz de resolver cualquier caso del problema en un tiempo acotado superiormente por ct(n) segundos, donde n es el tamaño del caso considerado. 
	
	\bigskip
	\textcolor{Azul}{Además, no habrá unidad para expresar el tiempo de ejecución de un algoritmo. Usaremos una constante (oculta) para acumular dichos factores relativos.}
	
	\bigskip
	\subsection{Notación Asintótica}
	Para comparar los algoritmos empleando los tiempos de ejecución y las constantes ocultas se emplea la denominada \textbf{notación asintótica}. Esta notación sirve para capturar las conductas de funciones para valores altos de x.
	
	\begin{teorema}
		\textbf{Definición formal para $O$}
		
		Sean $f$ y $g$ funciones definidas de $\mathbb{N} \rightarrow \Re^+$. Si dice que $f$ es de orden $g$, $O(g(n))$, si existen dos constantes positivas $C$ y $k$ tales que \[\forall n\geq k,f(n) \leq C\cdot g(n)\]
	\end{teorema}

	\textcolor{Lila}{Para probar que no pueden existir constantes C, k tales que pasado k, siempre se verifique la in-ecuación, podemos usar límites.}
	Por ejemplo:
	
	
\end{document}